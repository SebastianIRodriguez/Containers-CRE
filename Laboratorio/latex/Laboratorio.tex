
\documentclass[11pt]{article} % use larger type; default would be 10pt

\usepackage[utf8]{inputenc}

\usepackage[spanish, english]{babel}

%%% PAGE DIMENSIONS
\usepackage{geometry} % to change the page dimensions
\geometry{a4paper} % or letterpaper (US) or a5paper or....
% \geometry{margin=2in} % for example, change the margins to 2 inches all round
% \geometry{landscape} % set up the page for landscape
%   read geometry.pdf for detailed page layout information

\usepackage{graphicx}

\usepackage{minted}

% \usepackage[parfill]{parskip} % Activate to begin paragraphs with an empty line rather than an indent

%%% PACKAGES
\usepackage{booktabs} % for much better looking tables
\usepackage{array} % for better arrays (eg matrices) in maths
\usepackage{paralist} % very flexible & customisable lists (eg. enumerate/itemize, etc.)
\usepackage{verbatim} % adds environment for commenting out blocks of text & for better verbatim
\usepackage{subfig} % make it possible to include more than one captioned figure/table in a single float
% These packages are all incorporated in the memoir class to one degree or another...

%%% HEADERS & FOOTERS
\usepackage{fancyhdr} % This should be set AFTER setting up the page geometry
\pagestyle{fancy} % options: empty , plain , fancy
\renewcommand{\headrulewidth}{0pt} % customise the layout...
\lhead{}\chead{}\rhead{}
\lfoot{}\cfoot{\thepage}\rfoot{}

%%% SECTION TITLE APPEARANCE
\usepackage{sectsty}
\allsectionsfont{\sffamily\mdseries\upshape} % (See the fntguide.pdf for font help)
% (This matches ConTeXt defaults)

%%% ToC (table of contents) APPEARANCE
\usepackage[nottoc,notlof,notlot]{tocbibind} % Put the bibliography in the ToC
\usepackage[titles,subfigure]{tocloft} % Alter the style of the Table of Contents
\renewcommand{\cftsecfont}{\rmfamily\mdseries\upshape}
\renewcommand{\cftsecpagefont}{\rmfamily\mdseries\upshape} % No bold!

%%% END Article customizations

%%% The "real" document content comes below...

\title{Laboratorio de Containers}
\author{Ferreyra, Martinez Castro, S. Rodríguez}
\date{Junio, 2019} % Activate to display a given date or no date (if empty),
         % otherwise the current date is printed 

\begin{document}
\maketitle

\section{Laboratorio de Docker}

Esta es una primera aproximación a docker donde primero crearemos un container individual desde una imagen bajada desde el servidor de docker, oportunamente llamado DockerHub, para luego crear nuestro propia imagen con el codigo provisto y crear un container con este.

\subsection{Levantar un container para ejecutar una tarea}

Vamos a correr el comando ls desde un container con alpine que es una distribucion de linux para ver los contenidos de la raiz en el container.

Para eso vamos a correr 'docker container run alpine ls'.

Ese ultimo comando contiene varias cosas. 'docker' es el comando principal que se escribe para pasarle instrucciones a docker. Una de esas instrucciones es 'container' que permite administrar los containers y dentro de esa instruccion podemos especificar una subinstruccion para administrarlos que es 'run'.

'run' basicamente corre una instruccion en un nuevo container. De esta forma 'run' simultaneamente crea un nuevo contenedor con la imagen que le pasamos como parametro 'alpine' y corre una instruccion que le pasamos como segundo parametro, en este caso 'ls'.


Luego de correr este comando vamos a ver que docker bajo la imagen oficial desde la pagina y que creo el container para por ultimo correr el comando ls y mostrar la estructura del directorio en la raiz del container.

Podemos ver corriendo 'docker container ls --all' que nuestro container sigue vivo y con un estado que nos indica que ya no se esta utilizando. Ademas, podemos ver que nuestro  container tiene un id y una imagen asociados; en este caso la imagen es alpine.

\subsection{Correr un container interactivo de Ubuntu}

Para este ejemplo vamos a correr una sesión de bash desde un container levantado con una imagen de ubuntu. Para eso vamos a utilizar nuevamente el comando 'run' del ejemplo anterior con un par de parámetros extras.

Vamos a correr 'docker container run --interactive --tty --rm ubuntu bash'. 

Este comando utiliza los parámetros 'interactive', 'tty' y 'rm'. Estos parámetros en orden son:
\begin{itemize}

	\item
    interactive relaciona el input producido desde el teclado del host con la entrada del container para la sesión de bashdocker classroom
    
    \item
    ity muestra, como estamos acostumbrados, el usuario actual tanto como el directorio haciendolo mas parecido aun a una terminal corriente
    
    \item
    rm elimina el container luego de terminar la sesion. De esta forma no se vera mas al correr 'docker container ls --all'
\end{itemize}

Luego de eso podemos correr un par de comandos sobre el nuevo container que instanciamos:
\begin{itemize}
	\item    
    'ls /' nos mostrara lo mismo que previamente hicimos para alpine
    
    \item  
    'ps aux' nos muestra todos los procesos corriendo actualmente en el contenedor
    
    \item  
    'cat /etc/issue' nos va a mostrar la distribución de linux sobre la que estamos
\end{itemize}

Este ultimo ejemplo es especialmente util ya que de esta manera podemos ingresar a un contenedor, configurar todo el entorno para nuestra aplicacion, administrar las dependecias requeridas y testear la aplicacion; y una vez que tenemos el contenedor como queremos podemos generar una imagen desde este y de esta forma distribuir libremente la imagen para que puedan utilizar tu aplicacion y tener el mismo container.

\subsection{Correr un contenedor en background}

Esta va a ser la manera en que vamos a correr la mayoría de los containers.\\
Vamos a correr un container con MySQL para luego explicar las opciones utilizadas:\\ \\
		\texttt{docker container run} \textbackslash \\
		\texttt{--detach} \textbackslash \\
		\texttt{--name mydb} \textbackslash \\
		\texttt{-e MYSQL\_ROOT\_PASSWORD=1234} \textbackslash \\
		\texttt{mysql:latest}
	\\

En el comando anterior, ademas de añadir una etiqueta a la imagen que queriamos correr (mysql:latest, que nos trae la ultima version de MySQL), utilizamos opciones nuevas, que son:
\begin{itemize}
	\item 	
 	'detach', pone a correr el servidor en background por lo tanto no se conseguira ningun retorno de simplemente ejecutarlo, sino que deberemos acceder de alguna forma a este para utilizarlo.
 	
 	\item
	'name' le asigna un nombre personalizado al contenedor. Hasta ahora los nombres eran asignados aleatoriamente.
	
	\item
	'e' configura la variable que contiene la contraseña de la base de datos.

\end{itemize}

Podemos ver que en la ultima linea nos devuelve una combinacion de numeros y letras que corresponde al hash del container, pero aparte de eso, no imprime nada.

Para ver que esta funcionando podemos correr un comando 'docker container ls'. Lo siguiente es correr algun comando sobre este contenedor:\\ \\
	\texttt{docker exec -it mydb} \textbackslash \\
	\texttt{mysql --user=root --password=\$MYSQL\_ROOT\_PASSWORD --version}\\

En este comando vemos que se usa exec en vez de container, y lo que esto indica es que exec va a correr el comando que se le pase sobre un servidor que ya esta activo (a diferencia de run que levanta el container a la vez). 

Este comando nos va a mostrar la version de MySQL y va a concluir, pero aun asi no va a matar el container. 

Podriamos probar hacer de otra manera esto corriendo primero una consola desde el contenedor y luego ejecutando el mismo comando de antes. Para eso usamos:\\ \\
	\texttt{docker exec -it mydb sh} \\

Ahora estamos en una consola dentro del contenedor, por lo cual podemos correr: \\ \\
	\texttt{mysql --user=root --password=\$MYSQL\_ROOT\_PASSWORD --version} \\

Y vemos que da el mismo resultado.. Por ultimo para cerrar la sesion de consola corremos: exit


\subsection{Construir una imagen y correr un container personalizado}

En esta actividad vamos a construir una imagen de Docker usando un Dockerfile (que no es mas que un archivo de configuracion parecido a composer) y luego compilar esta imagen para levantar un container que contenga esta imagen. 

Para eso vamos a entrar en la carpeta de proyecto1 para ver el Dockerfile que armamos como ejemplo para esta actividad.

Al ver el Dockerfile, podemos encontrar:
\begin{itemize}
	\item
	FROM: especifica la imagen base que vamos a utilizar para el contenedor, que puede ser desde un MySQLServer hasta una instancia de linux en la que querramos correr comandos.

	\item
	COPY: que va a copiar los archivos presentes en el directorio dentro del container para asi utilizarlos durante la ejecucion de tareas.
	Copy espera primero la direccion de un archivo presente en la computadora y luego la direccion donde va a copiarlo dentro del container.
	
	\item	
	EXPOSE: documenta los puertos que la aplicacion va a necesitar. Mas adelante vamos a ver de que manera se pueden vincular estos puertos a puertos reales de nuestra computadora.
	
	\item
	CMD especifica que comando correr cuando el contenedor se levanta desde la imagen. Esta opcion se utilizar para configurar el contenedor antes de ingresar en el.
\end{itemize}


Luego, solo nos queda compilar la imagen para luego ejecutarla.. Para eso vamos a correr el comando: \\ \\
	\texttt{docker image build --tag mi\_app:1.0 .} \\

En este comando se utiliza el gestor de imagenes para construir una nueva imagen a la cual le otorgamos un nombre personalizado con la opcion --tag y le decimos que incluya el directorio actual con el .

Ademas al nombre de la imagen le agregamos un numero de version luego de los dos puntos para generar una version nueva mas adelante.

Despues de correrlo vemos que nos imprime todos los pasos que va siguiendo y por ultimo nos indica que la imagen se construyo correctamente.

Ahora podemos correr un 'docker container run' igual que antes usando la tag que le asignamos previamente a la imagen para correr este container personalizado.. Corremos: \\ \\
	\texttt{docker container run} \textbackslash \\
	\texttt{--detach} \textbackslash \\
	\texttt{--publish 80:80} \textbackslash \\
	\texttt{--name mi\_container} \textbackslash \\
	\texttt{mi\_app:1.0} \\ \\

Las novedades de este comando incluyen la opcion --publish que como se habia comentado antes, va a servir para vincular puertos reales de nuestra computadora con puertos publicados del contenedor y asi realizar pedidos al contenedor desde fuera.

Ahora podemos ver que nuestro contenedor esta funcionando escribiendo la direccion 'localhost:80' en un navegador y ver que se desplego tanto la pagina como el servidor.

Ahora que vimos que esta funcionando vamos a eliminarlo para pasar a una version mas dinamica del container. Vamos a correr:\\ \\
	\texttt{docker container rm --force mi\_app:1.0} \\

\subsection{Modificar un contenedor que esta en ejecución}

Al desarrollar, no es conveniente parar y volver a iniciar el contenedor cada vez que se realizan cambios. En eso esta centrado el siguiente ejercicio, donde vamos a montar el directorio con el codigo fuente dentro del contenedor. De esta forma, cualquier cambio efectuado en el directorio de nuestra computadora se va a reflejar directamente sobre el container, cambiando nuestra aplicacion automaticamente.

Para eso vamos a utilizar una 'bind mount' y para correr nuestro contenedor incluyendo esta opcion vamos a incluir la bandera --mount all comando container run de antes: \\ \\
	\texttt{docker container run} \textbackslash \\
	\texttt{--detach} \textbackslash \\
	\texttt{--publish 80:80} \textbackslash \\
	\texttt{--name mi\_container} \textbackslash \\
	\texttt{--mount type=bind,source=\"\$(pwd)\",target=/usr/share/nginx/html} \textbackslash \\
	\texttt{mi\_app:1.0} \\ \\

La bandera mount incluye varias variables que indican:
\begin{itemize}
	
	\item
	type le indica el tipo de montura que va a utilizar; en este caso una bind mount

	\item	
	source le indica que carpeta quiere montar
	
	\item	
	target indica donde va a montar la carpeta, con una direccion dentro del container
\end{itemize}

Al establecer un bind mount ahora podemos cambiar el contenido de nuestro archivo index.html y ver los cambios reflejados en la pagina, accediendo nuevamente a la direccion en la que el servidor esta corriendo.

Para hacer esta parte mas veloz ya se provee un archivo index-new.html con un cambio de color para la pagina el cual vamos a copiar con el nombre index.html; reescribiendolo para ver los cambios en la pagina.

Vamos a correr: \\ \\
	\texttt{cp index-new.html index.html} \\

Y ahora si vamos nuevamente a la direccion del localhost junto con el puerto del container, o recargamos la pagina en caso de que no la hayamos cerrado veremos los cambios.

Ahora que hemos probado las caracteristicas de los bind mount podemos probar a correr nuevamente el container sin el bind mount para ver si se mantienen los cambios.

Vamos a ejecutar los siguientes comandos: \\ \\
	\texttt{docker rm --force mi\_container} \\ \\
	\texttt{docker container run } \textbackslash \\
	\texttt{--detach } \textbackslash \\
	\texttt{--publish 80:80 } \textbackslash \\
	\texttt{--name mi\_container } \textbackslash \\
	\texttt{mi\_app:1.0} \\ \\

Y ahora si probamos a recargar la pagina veremos que esta devuelta en su estado original. Esto es devido a que los bind mount no afectan a la imagen en si sino al contenido del container y cada vez que se genere un nuevo container con la imagen nuestra va a tener el aspecto original que le otorgamos.

Nuevamente, vamos a detener el container para el siguiente paso: \\ \\
	\texttt{docker rm --force mi\_container} \\ \\

\subsection{Actualizar la imagen de docker}

Para hacer persistentes los cambios efectuados localmente en una imagen tendremos que generar una nueva version de la imagen.

Para eso podemos reutilizar el Dockerfile con el que construimos la version anterior, la unica diferencia es que ahora va a copiar el index.html cambiado.

Vamos a correr: \\ \\
	\texttt{docker image build --tag mi\_app:2.0 .} \\ \\

Y ya tenemos nuestra nueva version de la imagen. Podemos revisar todas las imagenes en el sistema hasta ahora corriendo: \\ \\
	\texttt{docker image ls} \\ \\

Ahora podemos probar la nueva version de nuestra aplicacion con un comando anterior: \\ \\
	\texttt{docker container run}  \textbackslash \\
	\texttt{--detach}  \textbackslash \\
	\texttt{--publish 80:80} \textbackslash \\
	\texttt{--name mi\_acontainer}  \textbackslash \\
	\texttt{mi\_app:2.0} \\ \\

Y recargando la pagina podremos ver la nueva version de nuestra pagina corriendo desde el container sin bind mounts. 

Ademas, podemos poner en simultaneo la otra version de la pagina corriendo, teniendo como precaucion no seleccionar el mismo puerto para los dos containers.

Vamos a correr: \\ \\
	\texttt{docker container run}  \textbackslash \\
	\texttt{--detach}  \textbackslash \\
	\texttt{--publish 8080:80}  \textbackslash \\
	\texttt{--name mi\_container\_antiguo}  \textbackslash \\
	\texttt{mi\_app:1.0} \\ \\

Ademas de asignar el puerto 8080 de nuestra computadora para acceder, hay que asignarle un nombre distinto a nuestro nuevo container para que docker los pueda diferenciar.

Ahora accediendo a localhost:8080 podemos ver la version antigua de nuestra aplicacion.



\section{Laboratorio de Kubernetes}


\end{document}

\documentclass{beamer}


\mode<presentation>
{
  \usetheme{Copenhagen}

  \setbeamercovered{transparent}
}


\usepackage[spanish, english]{babel}
% or whatever

\usepackage[utf8]{inputenc}

\usepackage{times}
\usepackage[T1]{fontenc}
% Or whatever. Note that the encoding and the font should match. If T1
% does not look nice, try deleting the line with the fontenc.


\title{Containers}

\author[S. I. Rodríguez]{Sebastian I. Rodríguez}


\institute
{
  Instituto Politécnico Superior Gral. San Martín
}

\date{Junio 2019}


% If you have a file called "university-logo-filename.xxx", where xxx
% is a graphic format that can be processed by latex or pdflatex,
% resp., then you can add a logo as follows:

% \pgfdeclareimage[height=0.5cm]{university-logo}{university-logo-filename}
% \logo{\pgfuseimage{university-logo}}



% Delete this, if you do not want the table of contents to pop up at
% the beginning of each subsection:
\AtBeginSection[]
{
  \begin{frame}<beamer>{Outline}
    \tableofcontents[currentsubsection]
  \end{frame}
}


% If you wish to uncover everything in a step-wise fashion, uncomment
% the following command: 

%\beamerdefaultoverlayspecification{<+->}


\begin{document}

\begin{frame}
  \titlepage
\end{frame}

\begin{frame}
  \frametitle{Índice}
  \tableofcontents
  % You might wish to add the option [pausesections]
\end{frame}

\section{Introducción}

\subsection{Definición de Container?}

\begin{frame}{Qué es un Container?}

  \begin{itemize}
  \item
    Es una unidad de software que contiene un programa y todas las dependencias necesarias para su funcionamiento.
  \item
   Introduce una abstracción que garantiza el funcionamiento del programa independientemente de las características del ambiente en el que se encuentre el container.
  \item 
  Aisla al programa del entorno en que se ejecuta.
  \end{itemize}
\end{frame}

\subsection{Container vs Máquina Virtual}
\subsection{Qué es Docker?}

\begin{frame}{Make Titles Informative.}

  You can create overlays\dots
  \begin{itemize}
  \item using the \texttt{pause} command:
    \begin{itemize}
    \item
      First item.
      \pause
    \item    
      Second item.
    \end{itemize}
  \item
    using overlay specifications:
    \begin{itemize}
    \item<3->
      First item.
    \item<4->
      Second item.
    \end{itemize}
  \item
    using the general \texttt{uncover} command:
    \begin{itemize}
      \uncover<5->{\item
        First item.}
      \uncover<6->{\item
        Second item.}
    \end{itemize}
  \end{itemize}
\end{frame}


\subsection{Second Subsection}

\begin{frame}{Make Titles Informative.}
\end{frame}

\begin{frame}{Make Titles Informative.}
\end{frame}



\section*{Summary}

\begin{frame}{Summary}

  % Keep the summary *very short*.
  \begin{itemize}
  \item
    The \alert{first main message} of your talk in one or two lines.
  \item
    The \alert{second main message} of your talk in one or two lines.
  \item
    Perhaps a \alert{third message}, but not more than that.
  \end{itemize}
  
  % The following outlook is optional.
  \vskip0pt plus.5fill
  \begin{itemize}
  \item
    Outlook
    \begin{itemize}
    \item
      Something you haven't solved.
    \item
      Something else you haven't solved.
    \end{itemize}
  \end{itemize}
\end{frame}


\end{document}


